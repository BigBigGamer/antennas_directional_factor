\documentclass[tikz]{standalone}

\usepackage[T2A]{fontenc}
\usepackage[utf8x]{inputenc}
\usepackage[russian]{babel}
% \usepackage[utf8x]{inputenc}
\usepackage{amsmath}
\usepackage{amssymb}

\usepackage{cmap,pgfplots,pgfplotstable}
\usetikzlibrary{arrows,calc}
\pgfplotsset{compat=newest}
\usepgfplotslibrary{polar}
\usepackage[outline]{contour}
\begin{document}
% \pgfplotstableread{data1-ris10.tsv}\mytable 
	\begin{tikzpicture}

           \begin{polaraxis}[
            height=5cm,
			width=5cm,   
			scale=1,
            ticks = none,
            major grid style={
				line width=0.5pt, 	% толщина основных линий сетКи
				% draw=black!50, 		% цвет основных линий сетКи: 50% черного (80% Белого) 
				draw=none,
			},
			%
			minor grid style={
				line width=0.5pt, 	% толщина Промежуточных линий сетКи
				% draw=black!20,		% цвет Промежуточных линий сетКи
				draw=none,
            },
            xmax = 180,
            ymax = 1,
            enlargelimits=true,            
            x axis line style ={draw=none},
            y axis line style ={draw=none},
           ]
            \addplot+[color =black, mark=none,domain=0:180,samples=600] 
                {sin(3*x)^2*sin(x)};
            % \node[align = center] at (60,1) {\scriptsize Главный \\ \scriptsize лепесток};
			% \node[align = center] at (150,1) {\scriptsize Побочные \\ \scriptsize лепестки};
            
                 
            % equivalent to (x,{sin(..)cos(..)}), i.e.
            % the expression is the RADIUS
            \end{polaraxis}  

	\end{tikzpicture}	
\end{document}