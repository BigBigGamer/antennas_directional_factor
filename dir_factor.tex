\input{text/diss}

\begin{document}

\def\labauthors{Виноградов И.Д., Шиков А.П.}
\def\labgroup{430}
\def\labnumber{1}
\def\labtheme{Определение коэффициента направленного действия рупорной антенны}
\input{text/titlepage}

{\bfseries Цель работы:} 
Нахождение коэффициента направленного действия пирамидальной рупорной антенны с помощью так называемого зеркального
метода (метод Парселла).
\section{Теоритическая часть}

Антенна — устройство, предназначенное для излучения или приема волн (в нашем случае — электромагнитных). Одна из
важнейших функций антенны состоит в формировании излучения с определенными направленными свойствами. Основными
характеристиками направленности антенны являются диаграмма направленности (ДН) по амплитуде или по мощности, коэффициент
 направленного действия (КНД) и коэффициент усиления (КУ). Напомним, как вводятся эти характеристики.

\textit{Диаграмма направленности} по амплитуде есть угловое распределение амплитуды поля излучения, т.е. зависимость этой
амплитуды от полярного $\theta $ и азимутального $ \varphi $ углов при фиксированном расстоянии $r$ от антенны. \textit{Диаграмма направленности
по мощности} есть угловое распределение мощности излучения в единицу телесного угла $P(\theta, \varphi)=r^{2} S_{r}(r, \theta, \varphi)$,
где $S_{r}$ — радиальная компонента вектора Пойнтинга на достаточно большом расстоянии $r$ от антенны. Представляется 
удобным использование (наряду с абсолютной) нормированной диаграммы направленности $F(\theta,\varphi) = P(\theta,\varphi)/P(\theta_m,\varphi_m)$,
где $P(\theta_m,\varphi_m)$ — мощность, излучаемая в единичный телесный угол в направлении главного максимума $(\theta_m,\varphi_m)$ диаграммы
направленности. Диаграмму направленности изображают графически либо в виде <<объемной>>, рельефной картины, где по
каждому угловому направлению $(\theta,\varphi)$ откладывается величина, пропорциональная амплитуде поля излучения или излучаемой 
мощности (см. рис. 1а), либо с помощью плоской развертки отдельных, чаще всего двух ортогональных сечений, проходящих 
через направление главного максимума и векторы электрического \textbf{Е} и магнитного \textbf{Н} полей (см. рис.16). Поскольку основная
часть мощности, излучаемой направленной антенной, сосредоточена, как правило, в главном лепестке, то весьма показательной
представляется его угловая ширина, определяемая обычно по уровню половинной мощности $ \left(\Delta \theta_{0,5}\right) $, а иногда и по нулевому (или минимальному) значению
$ \left(\Delta \theta_{0}\right) $, как показано на рис.1 б. Диаграмма направленности антенны, характерный размер $l$
излучающей апертуры которой порядка или больше длины излучаемой волны $\lambda$, окончательно формируется в зоне Фраунгофера,
определяемой соотношением 

\begin{equation}
    r>>\frac{l^2}{\lambda}
    \label{eq:1}
\end{equation}

\textit{Коэффициент направленного действия} D характеризует
выигрыш по  мощности в направлении максимального излучения вследствие направленности антенны. Он равен отношению 
мощности, излучаемой в единицу телесного угла в направлении максимума диаграммы направленности $P(\theta_m,\varphi_m)$, к средней 
мощности $Р_{cp} = Р_{\text{изл}} /(4\pi)$, излучаемой антенной по всем направлениям, т.е. $ D=4 \pi
P\left(\theta_{m}, \varphi_{m}\right) / P_{\text{изл}} $, где $P_{\text{изл}} $ — полная излучаемая мощность:
$$  P_{\text{изл}} = \int_0^{2\pi}d\varphi \int_0^\pi P( \theta,\varphi ) \sin{\theta}d \theta. $$

Таким образом, имеем:
\begin{equation}
    D=\frac{4 \pi P\left(\theta_{m}, \varphi_{m}\right)}{\int_{0}^{\pi} d \varphi \int P(\theta, \varphi) \sin \theta d \theta}
    \label{eq:2}    
\end{equation}

\textit{Коэффициент усиления} G определяется как произведение КНД на коэффициент полезного действия (КПД) антенны $\eta$
(или, точнее, всего антенного тракта):

\begin{equation}
    G = D\eta
    \label{eq:3}    
\end{equation}

Этот последний коэффициент в свою очередь есть отношение полной мощности $P_{\text{изл}}$, излучаемой антенной, к полной
мощности $P_{\text{подв}}$, подводимой к антенне, т.е.
\begin{equation}
    \eta =\frac{P_{\text{изл}}}{P_{\text{подв}}} = \frac{\int_{0}^{2 \pi} d \varphi \int_{0}^{\pi} P(\theta, \varphi) \sin \theta d \theta}{P_{\text{подв}}}
    \label{eq:4}    
\end{equation}

В силу принципа взаимности ДН и КНД антенны при ее работе в режиме передачи и в режиме приема совпадают.


Для адекватного описания \textit{приемной антенны} вводятся некоторые дополнительные характеристики. Одна из основных таких характеристик — эффективная площадь приема антенны $А$.


\textit{Эффективная площадь} приема $А$ определяется как отношение полной принимаемой антенной мощности $P_{\text{пр}}$ к плотности потока падающего
излучения $S_n$ в месте расположения антенны:
\begin{equation}
    A = \frac{P_{np}}{S_n}
    \label{eq:5}
\end{equation}
Как показано в \ref{eq:1},\ref{eq:2}, величины $A$ и $D$ связаны соотношением
\begin{equation}
    A = \frac{\lambda^2}{4\pi}D.
    \label{eq:6}
\end{equation}

Цель настоящей работы заключается в экспериментальном определении КНД пирамидальной рупорной антенны с помощью так 
называемого зеркального метода (метода Парселла) и сравнении измеренного значения с рассчитанным теоретически. 
Зеркальный метод опирается на использование идеально (зеркально) отражающей плоской поверхности, расположенной в зоне 
Фраунгофера и ориентированной параллельно излучающей апертуре (см. рис. 2).


Согласно методу изображений отыскание отраженного поля, поступающего в антенну, сводится к нахождению поля, 
принимаемого от аналогичной зеркальной относительно отражающей плоскости излучающей антенны (рис. 2). В результате 
последовательного пересчета имеем: мощность, излучаемая гипотетической зеркальной антенной в единицу телесного угла 
в направлении на реальную антенну, равна $P_n = D Р_{\text{изл}}/4\pi$, откуда плотность потока энергии в месте приема 
$S_n = P_n/4X^2 = D P_{\text{изл}}/(16\pi X^2)$, где $X$ — расстояние между антенной и отражающей плоскостью; наконец, 
мощность, принимаемая антенной, равна $P_{np} =A S_n =A D P_{\text{изл}}/(16\pi X^2)$. С учетом \ref{eq:6} окончательно 
получаем
\begin{equation}
    \frac{P_{np}}{P_{\text{изл}}} = \frac{D^2\lambda^2}{64\pi^2X^2}
    \label{eq:7}
\end{equation}

отсюда интересующая нас величина $D$ представляется в виде 
\begin{equation}
    D = \frac{8/pi X}{\lambda}\sqrt{\frac{P_{np}}{P_{\text{изл}}}}
    \label{eq:8}
\end{equation}
% \begin{center}
%     \begin{minipage}[t]{0.49\linewidth}
%         \includegraphics[width=\linewidth]{R1.png} 
%         \label{fig:1}
%         \vspace{-32pt}
%         \captionof{figure}{} 
%     \end{minipage}
%     \begin{minipage}[t]{0.49\linewidth}
%         \includegraphics[width=\linewidth]{1.jpg} 
%         \label{fig:2}
%         \vspace{-32pt}
%         \captionof{figure}{} 
%     \end{minipage}
% \end{center}



\newpage
\section{Экспериментальная часть}



\end{document}